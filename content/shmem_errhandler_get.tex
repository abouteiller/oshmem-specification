\apisummary{
    This routine obtains the handler currently invoked when reporting an error.
}

\begin{apidefinition}

\begin{Csynopsis}
void shmem_errhandler_get(int errcode, void (*errhandler_fn)(int errcode, int origin_pe, void* user_context), void* user_context);
\end{Csynopsis}

\begin{Fsynopsis}
INT ERRCODE
EXTERNAL ERRHANDLER_FN
POINTER (USER_CONTEXT, POINTEE)
SHMEM_ERRHANDLER_GET(ERRCODE, ERRHANDLER_FN, USER_CONTEXT)
\end{Fsynopsis}

\begin{apiarguments}
    \apiargument{IN}{errcode}{The error code to which the error handler is
        attached.}
    \apiargument{OUT}{errhandler\_fn}{The error handler function currently set.}
    \apiargument{OUT}{user\_context}{The application parameter passed to the error handler
        when it is invoked with this error code.}
\end{apiarguments}

\apidescription{
    This routine gets the error handler and application parameter set for 
    the error code passed
    as argument. When the returned \textit{errhandler\_fn} is one of the 
    predefined error handlers (i.e. \textit{SHMEM\_ERRHANDLER\_GEXIT},
    \textit{SHMEM\_ERRHANDLER\_BREAK}, and \textit{SHMEM\_ERRHANDLER\_GBREAK}), the \textit{user\_context} argument is
    set to NULL.
}

\apireturnvalues{
    None.
}

\apinotes{
    None.
}

\end{apidefinition}
