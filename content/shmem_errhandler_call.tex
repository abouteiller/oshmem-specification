\apisummary{
    This routine calls an error handler.
}

\begin{apidefinition}

\begin{Csynopsis}
void shmem_errhandler_call(int errcode, void (*errhandler_fn)(int errcode, int origin_pe, void* user_context));
\end{Csynopsis}

\begin{Fsynopsis}
INT ERRCODE
EXTERNAL ERRHANDLER_FN
SHMEM_ERRHANDLER_CALL(ERRCODE, ERRHANDLER_FN)
\end{Fsynopsis}

\begin{apiarguments}
    \apiargument{IN}{errcode}{The error code to pass to the error handler.}
    \apiargument{IN}{errhandler\_fn}{The error handler function to invoke.}
\end{apiarguments}

\apidescription{
    This routine calls the error handler \textit{errhandler\_fn} for
    the error code \textit{errcode}, with the local \ac{PE}, and the implicit
    \textit{user\_context} (as set during registration) as arguments. Either predefined
    error handlers (i.e. \textit{SHMEM\_ERRHANDLER\_GEXIT},
    \textit{SHMEM\_ERRHANDLER\_BREAK}, and \textit{SHMEM\_ERRHANDLER\_GBREAK}),
    or user defined functions may be passed in \textit{errhandler\_fn}.
    It is valid to call an error handler from within an error handler function.
    
    When an user provided error handler function returns, it has the same
    effect as if it had called \textit{SHMEM\_ERRHANDLER\_BREAK} as its last
    statement.
}

\apireturnvalues{
    None.
}

\apinotes{
    None.
}

\end{apidefinition}
